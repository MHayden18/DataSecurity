\documentclass{article}
\usepackage{graphicx}
\usepackage{alltt}
\usepackage{amsmath}
\usepackage{amsfonts}
\usepackage{bigstrut}
\usepackage{enumerate}
\usepackage{fancyhdr}
\usepackage[top=.75in, bottom=.95in, left=.75in, right=.75in]{geometry}
\usepackage{float}
\usepackage{lastpage}
\usepackage{tikz}
\usepackage[latin1]{inputenc}
\usepackage{color}
\usepackage{array}
\usepackage{longtable}
\usepackage{calc}
\usepackage{multirow}
\usepackage{hhline}
\usepackage{ifthen}
\usepackage{listings}
\usepackage{circuitikz}
%\lstset{language=VHDL,basicstyle=\ttfamily}
\definecolor{mygreen}{rgb}{0,0.6,0}
\definecolor{mygray}{rgb}{0.5,0.5,0.5}
\definecolor{mymauve}{rgb}{0.58,0,0.82}
\lstset{ %
  backgroundcolor=\color{white},   % choose the background color; you must add \usepackage{color} or \usepackage{xcolor}; should come as last argument
  basicstyle=\normalsize,        % the size of the fonts that are used for the code
  breakatwhitespace=false,         % sets if automatic breaks should only happen at whitespace
  breaklines=true,                 % sets automatic line breaking
  captionpos=b,                    % sets the caption-position to bottom
  commentstyle=\color{mygreen},    % comment style
  deletekeywords={...},            % if you want to delete keywords from the given language
  escapeinside={\%*}{*)},          % if you want to add LaTeX within your code
  extendedchars=true,              % lets you use non-ASCII characters; for 8-bits encodings only, does not work with UTF-8
  frame=single,	                   % adds a frame around the code
  keepspaces=true,                 % keeps spaces in text, useful for keeping indentation of code (possibly needs columns=flexible)
  keywordstyle=\color{blue},       % keyword style
  language=VHDL,                   % the language of the code
  morekeywords={*,...},            % if you want to add more keywords to the set
  numbers=left,                    % where to put the line-numbers; possible values are (none, left, right)
  numbersep=5pt,                   % how far the line-numbers are from the code
  numberstyle=\tiny\color{mygray}, % the style that is used for the line-numbers
  rulecolor=\color{black},         % if not set, the frame-color may be changed on line-breaks within not-black text (e.g. comments (green here))
  showspaces=false,                % show spaces everywhere adding particular underscores; it overrides 'showstringspaces'
  showstringspaces=false,          % underline spaces within strings only
  showtabs=false,                  % show tabs within strings adding particular underscores
  stepnumber=2,                    % the step between two line-numbers. If it's 1, each line will be numbered
  stringstyle=\color{mymauve},     % string literal style
  tabsize=2,	                   % sets default tabsize to 2 spaces
  title=\lstname                   % show the filename of files included with \lstinputlisting; also try caption instead of title
}
\floatstyle{plain}
\restylefloat{figure}
\pagestyle{fancy}
\fancyhead{}
\fancyfoot{}
\setlength{\headheight}{59.0pt}
\def\inputGnumericTable{}
\fancyhead[CO]{\textbf{Air Force Institute of Technology\\Department of Electrical and Computer Engineering\\
Data Security(CSCE 544)}\\ Homework \#5\\
 Due Date: \textbf{20-May-2019}\\Micah Hayden}
\lhead{\today}
\rhead{Page \thepage{} of \pageref{LastPage} }
\newlength\tindent
\setlength{\tindent}{\parindent}
\setlength{\parindent}{0pt}
\renewcommand{\indent}{\hspace*{\tindent}}
\begin{document}

\section{[10 points] Given the following RSA public-key: pk =\{e,n\} =\{5, 29623244235986089658629749102364587250074379396847895501789\\
74023998496502140571365918899417655751929\}}
\subsection{[8 points] Determine the prime numbers $p$ and $q$:}
The two prime numbers are shown below, factored using msieve:
\begin{verbatim}
p50 factor: 47720170418078602591676381343885005276759196131213
p50 factor: 62076987522164082179415623963823156278081468066333
\end{verbatim}

\subsection{[2 points] Compute Euler's Totient function $\phi(n)$:}
The totient function is shown below:
$$ \phi(n) = 2962324423598608965862974910236458725007437939684679753021033781313725410135263657757344576991554384 $$

\section{[30 points] Given prime numbers $p=315349$ and $q=259907$ and $e=5$:}
\subsection{[7 Points] Construct public key pk=\{e,n\}}
Using $n = p \cdot q$, this gives the following public key:
\begin{align*}
pk &= (5, 81961412543)
\end{align*}

\subsection{[7 Points] Determine Euler's totient function $\phi(n)$:}
Using $\phi(n) = (p-1) \cdot (q-1)$, this gives the following:
\begin{align*}
\phi(n) = 81960837288
\end{align*}

\subsection{[7 Points] Determine the private-key=\{d,n\}:}
Using $e \cdot d = 1 \mod \phi(n)$, gives the following key:
\begin{align*}
private \, key = (49176502373, \, 81961412543)
\end{align*}

\newpage
\subsection{[9 Points] Compute the cipher text for EACH of the following ASCII (8-bits) characters: \texttt{``The US Army will never control the ground under the sky, if the US Air Force does not control the sky over the ground.'' -- Col Gene \\Cirillo, USAF (Ret). }}
The ciphertext is shown below:
\lstinputlisting{Ciphertext.txt}

\section{[30 points] Given the following prime numbers, list all numbers $\alpha$ that can be used as generators in a cyclic group $\mathbb{Z}_p^{*}$}

\begin{enumerate}
 \item [1. ]41
 
 $\alpha \in$ [6, 7, 11, 12, 13, 15, 17, 19, 22, 24, 26, 28, 29, 30, 34, 35]
 
 \item [2. ]43
 
 $\alpha \in$ [3, 5, 12, 18, 19, 20, 26, 28, 29, 30, 33, 34]
 
 \item [3. ]71
 
 $\alpha \in$  [7, 11, 13, 21, 22, 28, 31, 33, 35, 42, 44, 47, 52, 53, 55, 56, 59, 61, 62, 63, 65, 67, 68, 69]
 
 \item [4. ]73
 
 $\alpha \in$ [5, 11, 13, 14, 15, 20, 26, 28, 29, 31, 33, 34, 39, 40, 42, 44, 45, 47, 53, 58, 59, 60, 62, 68]
 
 \item [5. ]541
 
 $\alpha \in$ [2, 10, 13, 14, 18, 24, 30, 37, 40, 51, 54, 55, 59, 62, 65, 67, 68, 72, 73, 77, 83, 86, 87, 91, 94, 96, 98, 99, 107, 113, 114, 116, 117, 126, 127, 128, 131, 132, 138, 150, 152, 153, 156, 158, 163, 176, 181, 183, 184, 195, 197, 199, 206, 208, 210, 213, 218, 220, 223, 224, 244, 248, 250, 257, 258, 259, 260, 261, 263, 267, 269, 270, 271, 272, 274, 278, 280, 281, 282, 283, 284, 291, 293, 297, 317, 318, 321, 323, 328, 331, 333, 335, 342, 344, 346, 357, 358, 360, 365, 378, 383, 385, 388, 389, 391, 403, 409, 410, 413, 414, 415, 424, 425, 427, 428, 434, 442, 443, 445, 447, 450, 454, 455, 458, 464, 468, 469, 473, 474, 476, 479, 482, 486, 487, 490, 501, 504, 511, 517, 523, 527, 528, 531, 539]
\end{enumerate}

\section{[30 points] Alice and Bob publicly agree to use a modulus $p=1999$ and a generator $\alpha=1994$. Alice chooses a secret integer $a= 1997$, and Bob chooses a secret integer $b =  2001$. Compute the shared secret integer $s$ using the Diffie-Hellman Key Exchange algorithm. }

The secret integer is  1983.

\section{[50 points] Alice and Bob publicly agree to use a modulus $p=2999$ and a generator $\alpha=161$. Alice sends  secret integer $a= 2341$ to Bob, and Bob sends secret integer $b =  192$. Compute the shared secret integer $s$ using the Diffie-Hellman Key Exchange algorithm. }

The secret integer is 2377.
The code for Problems 4 and 5 is shown in the attached Python scripts.

\end{document}

