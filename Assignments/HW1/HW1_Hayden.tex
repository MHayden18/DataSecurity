\documentclass{article}
\usepackage{graphicx}
\usepackage{alltt}
\usepackage{amsmath}
\usepackage{mathtools}
\usepackage{cancel}
\usepackage{amsfonts}
\usepackage{bigstrut}
\usepackage{enumerate}
\usepackage{fancyhdr}
\usepackage[top=.75in, bottom=.95in, left=.75in, right=.75in]{geometry}
\usepackage{float}
\usepackage{lastpage}
\usepackage{tikz}
\usepackage[latin1]{inputenc}
\usepackage{color}
\usepackage{array}
\usepackage{longtable}
\usepackage{calc}
\usepackage{multirow}
\usepackage{hhline}
\usepackage{ifthen}
\usepackage{listings}
\usepackage{circuitikz}
\usepackage{caption}
\definecolor{mygreen}{rgb}{0,0.6,0}
\definecolor{mygray}{rgb}{0.5,0.5,0.5}
\definecolor{mymauve}{rgb}{0.58,0,0.82}
\lstset{ %
  backgroundcolor=\color{white},   % choose the background color; you must add \usepackage{color} or \usepackage{xcolor}; should come as last argument
  basicstyle=\normalsize,        % the size of the fonts that are used for the code
  breakatwhitespace=false,         % sets if automatic breaks should only happen at whitespace
  breaklines=true,                 % sets automatic line breaking
  captionpos=b,                    % sets the caption-position to bottom
  commentstyle=\color{mygreen},    % comment style
  deletekeywords={...},            % if you want to delete keywords from the given language
  escapeinside={\%*}{*)},          % if you want to add LaTeX within your code
  extendedchars=true,              % lets you use non-ASCII characters; for 8-bits encodings only, does not work with UTF-8
  frame=single,	                   % adds a frame around the code
  keepspaces=true,                 % keeps spaces in text, useful for keeping indentation of code (possibly needs columns=flexible)
  keywordstyle=\color{blue},       % keyword style
  language=python,                  % the language of the code
  morekeywords={*,...},            % if you want to add more keywords to the set
  numbers=left,                    % where to put the line-numbers; possible values are (none, left, right)
  numbersep=5pt,                   % how far the line-numbers are from the code
  numberstyle=\tiny\color{mygray}, % the style that is used for the line-numbers
  rulecolor=\color{black},         % if not set, the frame-color may be changed on line-breaks within not-black text (e.g. comments (green here))
  showspaces=false,                % show spaces everywhere adding particular underscores; it overrides 'showstringspaces'
  showstringspaces=false,          % underline spaces within strings only
  showtabs=false,                  % show tabs within strings adding particular underscores
  stepnumber=2,                    % the step between two line-numbers. If it's 1, each line will be numbered
  stringstyle=\color{mymauve},     % string literal style
  tabsize=2,	                   % sets default tabsize to 2 spaces
  title=\lstname                   % show the filename of files included with \lstinputlisting; also try caption instead of title
}
\floatstyle{boxed}
\floatstyle{plain}
\restylefloat{figure}
\pagestyle{fancy}
\fancyhead{}
\fancyfoot{}
\setlength{\headheight}{59.0pt}
\def\inputGnumericTable{}
\fancyhead[CO]{\textbf{Air Force Institute of Technology\\Department of Electrical and Computer Engineering\\
 Cryptography and Data Security (CSCE-544) In Class Work Day \#3\newline \newline Name: Micah Hayden}}
\lhead{\today}
\rhead{Page \thepage{} of \pageref{LastPage} }
\newlength\tindent
\setlength{\tindent}{\parindent}
\setlength{\parindent}{0pt}
\renewcommand{\indent}{\hspace*{\tindent}}
\begin{document}
\subsection*{[100 Points] Modify the Caesar Shift Cipher Python Implementation Listed Such That Takes the Script Takes as Input a Binary File and Encrypts/Decrypts the Output to Another Binary File}
\begin{figure}[H]
\begin{lstlisting}
message = 'This is my secret message'
key = 11
mode = 'encrypt'
SYMBOLS='ABCDEFGHIJKLMNOPQRSTUVWXYZ abcdefghijklmnopqrstuvwxyz1234567890!?.'
translated=''
for symbol in message:
    if symbol in SYMBOLS:
        symbolIndex=SYMBOLS.find(symbol)
        if mode =='encrypt':
            translatedIndex=(symbolIndex+key)%len(SYMBOLS)
        elif mode == 'decrypt':
            translatedIndex=(symbolIndex-key)%len(SYMBOLS)
        translated+=SYMBOLS[translatedIndex]
    else:
        translated+=symbol
print(translated)
\end{lstlisting}
\vspace{-1cm}
\caption*{Sweigart, A.(2018). \emph{Cracking Odes with Python}. San Francisco, CA: No Starch Press}
\end{figure}

Below is the Python script I wrote for the caesar cipher, I utilized "Affine\_Cipher.py" as the test input to verify functionality.
\begin{figure}[H]
\begin{lstlisting}
print("Homework 1: byte-wise caesar cypher")

def Caesar_Bytewise(shift, inFile, outFile):
	with open(inFile, "rb") as f:
		with open(outFile, "wb") as output:
			byte = f.read(1)
			while byte != b"":
				# Convert byte into an integer value
				value = int.from_bytes(byte, byteorder='big')
				# Calculate the output value given the shift
				out_value = ( value + shift) % 256
				# Write the output value to the output file, as a byte
				output.write( out_value.to_bytes(1, byteorder='big', signed=False))
				byte = f.read(1)
			output.close()
		f.close()

def main():
	Caesar_Bytewise(100, "Affine_Cipher.py", "HW1_output.txt")
	
	# Used for testing:
	#Caesar_Bytewise(156, "HW1_output.txt", "HW1_outputVerification.txt")


if __name__ == "__main__":
	main()
\end{lstlisting}
\vspace{-1cm}
\caption*{Modified function for byte-wise caesar cipher, written to be portable to larger python script}
\end{figure}

\subsection*{[5 Points] Represent in binary using Two's complement (signed) - Calculators, Internet allowed}
Converting to Two's complement:  \newline
If positive: $2's \, Complement(x) = x $; if negative: $2's \, Complement(x) \, = \, x \oplus \mathtt{0xFF} + 1$

\begin{enumerate}
  \setlength{\itemsep}{20pt}%
  \item[a)]127: 
  $$ 127 = \boxed{0111 \, 1111} $$
  \item[b)]-128:
  $$ 128 = 1000 0000 \rightarrow 0111 1111 + 1 = \boxed{1000 0000} $$
  \item[c)]-1:
  $$ 1 = 0000 0001 \rightarrow 1111 1110 + 1 = \boxed{1111 1111} $$
\end{enumerate}

\subsection*{[5 Points] Represent the following decimal numbers in hexadecimal (unsigned) - Calculators, Internet allowed}
\begin{enumerate}
  \setlength{\itemsep}{20pt}
  \item[a)] 10:
  $$ 10 = \boxed{\mathtt{0xA}} $$
  \item[b)] 33:
  $$ 33 = 16 \cdot 2 + 1 = \boxed{\mathtt{0x21}} $$
  \item[c)] 120:
  $$ 120 = 16 \cdot 7 + 8 = \boxed{\mathtt{0x78}} $$
  \item[]
\end{enumerate} 
\subsection*{[5 Points] Find two different words (plainText$_1$, plainText$_2$) that are at least 4 characters long such that the cipher-texts produced using the Caesar Cipher algorithm are identical (cipherText$_1$ = cipherText$_2$) provided that they use two different keys (key$_1$, key$_2$). List the two plain-text words (plainText$_1$, plainText$_2$),  and their corresponding keys (key$_1$, and key$_2$). }

The below pairing was found by using a brute force approach with two online Caesar Cipher algorithms.  For the first, I chose a four letter word at random, and shifted it by some key $x$.  This produced plainText$_1 \rightarrow $ cipherText$_1$ with $key=x$.  I then used cipherText$_1$ as the input to the second Caesar Cipher, and checked each key to see if it produced a corresponding English word, let this key be $x'$.  This would indicate that $key_2 = 26 - x'$.

The words, cipher texts, and keys are as follows:

\begin{table}[h!]
\centering
\begin{tabular}{|c|c|c|} \hline
\textbf{Plain Text} & \textbf{Cipher Text} & \textbf{Key}  \\ \hline
cows & htbx & 5 \\ \hline
semi & htbx & 15 \\ \hline
\end{tabular}
\caption*{}
\end{table}

\subsection*{[10 Points] Fill in the blank (all from the book):}

\begin{enumerate}
  \setlength{\itemsep}{20pt}
  \item[a)] \underline{ Symmetric Encryption}  Participants share a secret key to encrypt and decrypt and decryption is the inverse operation of encryption.
  \item[b)]  \underline{ Asymmetric/Public Key Encryption} Participants share a public portion of the key, but a private portion is retained by a single participant for secure encryption and decryption and digital signatures.
  \item[c)] \underline{ Cryptography } The science of secret writing.
  \item[d)] \underline{ Cryptanalysis } The science and art of breaking the secure properties of crypto systems.
  \item[e)] \underline{ Moore's Law } The concept that computational power doubles every eighteen months so that brute force attacks against cryptosystems will be faster in the future.
  \item[f)] \underline{ Kerckhoffs' Principle } The concept that a cryptosystem should be secure even if the attacker knows all the details (but not the keys) of the targeted cryptosystem.
  \item[]
\end{enumerate}

\setcounter{section} {1}
\setcounter{subsection} {8}
\subsection{[25 Points] Compute $x$ as far as possible without a calculator. Where appropriate, make use of a smart decomposition of the exponent as shown in the example in Sect. 1.4.1:}


\begin{enumerate}
  \setlength{\itemsep}{10pt}%
  \item[a)]  $x = 3^2 \mod 13$
  $$ \boxed{x = 9} $$

  \item[b)] $x = 7^2 \mod 13$
  $$ x = 49 = 13 \cdot 3 + 10 \rightarrow \boxed{x = 10}$$
  \item[c)] $x = 7^{100} \mod 13$
  \begin{align*}
  x &= 7^{100} \mod 13 = (7^2)^{50} \mod 13 = 10^{50} \mod 13 	& 												\\
  	&														  	& 10^2 \mod 13 = 9								\\
  	&							  	  						  	& 10^3 \mod 13 = 10^2 \cdot 10 \mod 13 			\\
  	&														  	& 10^3 \mod 13 = 90 \mod 13 = 12				\\
  x &= 10^{50} \mod 13 = (10^2 \cdot 10^3)^{10} \mod 13 		& 												\\
  x &= (9 \cdot 12)^{10} \mod 13								& 9 \cdot 12 \mod 13 = 108 \mod 13 = 4 			\\
  x &= 4^{10} \mod 13 = (4^2)^5 \mod 13						  	& 4^2 \mod 13 = 3 \\
  x &= 3^5 \mod 13 = 3^2 \cdot 3^3 \mod 13					  	& 3^3 \mod 13 = 27 \mod 13 = 1 					\\
  x &= 9 \cdot 1 \mod 13 									  	& 												\\
  x &= 7^{100} \mod 13 = \boxed{9 = x}							&									
  \end{align*}
  
  \newpage
  \item[d)] This last problem is called a \emph{discrete logarithm} and points to a hard problem which we discuss in Chap. 8. The security of many public-key schemes is based on the hardness of solving the discrete logarithm for large numbers, e.g., with more than 1000 bits.\\$7\times x = 11 \mod 13$
  \begin{align*}
  7 \cdot x = & \, 11 \mod 13 \\
  7 \cdot 9 = & \, 63 \\
  			  & \, 63 \mod 13 = 11 \\
  \boxed{x = 9} &
  \end{align*} 
\end{enumerate}

\setcounter{subsection} {10}
\subsection{[25 Points] This problem deals with the affine cipher with the key parameters $\alpha = 7$, $\beta = 22$.}
I utilized a python script to solve this problem, and the code I wrote is included in Appendix A.
\begin{enumerate}
    \setlength{\itemsep}{10pt}%
    \item[a)] Decrypt the following text: \texttt{falszztysyjzyjkywjrztyjztyynaryjkyswarztyegyyj}
    
    The decrypted text is as follows:  \textbf{firstthesentenceandthentheevidencesaidthequeen}
    \item[b)] Who wrote the line?
    
    Lewis Caroll in \textit{Alice's Adventures Under Ground}
    \item[]
\end{enumerate}

\setcounter{subsection}{12}
\newpage
\subsection{[25 Points] In an attack scenario, we assume that the attacker Oscar manages somehow to provide Alice with a few pieces of plaintext that she encrypts.Show how Oscar can break the affine cipher by using two pairs of plaintext-ciphertext, $(x_1 ,y_1)$ and $(x_2 ,y_2)$. What is the condition for choosing $x_1$ and $x_2$?}

\textbf{Remark}:  Inpractice,such an assumption turns out to be valid for certain settings, e.g., encryption by Web servers, etc. This attack scenario is,thus,very important and is denoted as a \emph{chosen plaintext} attack.
\vspace{.5 cm}

One must select $x_1$ and $x_2$ such that $\mathbf{(x_2 - x_1) \mod m = 1 \rightarrow (x_2 - x_1) = m^{-1}}$.

The derivation of this condition is shown below, starting from the definition of the affine cipher: $(x \cdot \alpha + \beta) \mod m = y$, under the constraints: $1 \leq \alpha \leq m-1$ and $0 \leq \beta \leq m-1$.  
The constraints of $\alpha$ and $\beta$ imply the following:
\begin{align*}
\alpha \mod m = \alpha \hspace{.25\textwidth} \beta \mod m = \beta 
\end{align*}

\noindent \textbf{Derivation:}
\hrule

\begin{minipage}{.45\textwidth} %
 \begin{align*}
(x_1 \cdot \alpha + \beta) \mod m &= y_1 \\
(x_2 \cdot \alpha + \beta - x_1 \cdot \alpha - \beta) \mod m &= y_2 - y_1 \\
 \alpha \cdot (x_2 - x_1) \mod m &= y_2 - y_1 \\
\text{\textbf{Let $\mathbf{x_2-x_1 = m^{-1}}$, which yields:}}\\
  \alpha \cdot m^{-1} \mod m &= y_2 - y_1 \\
  \Aboxed{\alpha &= y_2 - y_1}  \\
  \\
\end{align*}
\end{minipage} %
\begin{minipage}{.5\textwidth}
\begin{align*}
&(x_2 \cdot \alpha + \beta) \mod m = y_2 \\
\text{Substitute the value $\alpha = y_2-y_1$:} & \\
&x_2 \cdot (y_2 - y_1) + \beta \mod m = y_2 \\
\text{Multiply both sides by $m^{-1}$:} & \\
& x_2 \cdot y_2 - x_2 \cdot y_1 + \beta = y_2 \cdot m^{-1} \\
& \beta = y_2 \cdot (x_2 - x_1) - x_2 \cdot y_2 + x_2 \cdot y_1  \\
& \Aboxed{\beta = x_2 \cdot y_1 - y_2 \cdot x_1 }
\end{align*}
\end{minipage}


Thus, by letting $x_2-x_1 = m^{-1}$, one arrives at the following system of equations, which can be solved for $\alpha$ and $\beta$, given values for $(x_1, y_1)$ and $(x_2, y_2)$.\footnote{Eq. \#2 includes $\mod m$ in case the difference $x_2 \cdot y_1 - y_2 \cdot x_1 > m$.  This follows from the equality $\beta \mod m = \beta$.}

\vspace{-.5 cm}
\begin{align*}
[1] \hspace{.5cm} \alpha = y_2 - y_1 \hspace{.25\textwidth} [2] \hspace{.5cm}\beta = x_2 \cdot y_1 - y_2 \cdot x_1 \mod m 
\end{align*}

\newpage
\subsection*{Appendix A:  Python Script}
\begin{minipage}[t]{1.05\textwidth}
\begin{lstlisting}
import string
import math

def ModularInverse(alpha):
	running = True
	inv_alpha = 1
	while running:
		if ( (alpha * inv_alpha) % 26 == 1):
			print( "Modular inverse of {0} is: {1}".format(alpha, inv_alpha) )
			running = False
			return inv_alpha
		else:
			inv_alpha += 1
			

def affineDecrypt(inv_alpha, beta, input):
	output = ""
	for letter in input:
		in_index = string.ascii_lowercase.index(letter)
		out_index = ( inv_alpha * (in_index - beta) ) % 26 
		output += string.ascii_lowercase[out_index]
	return output

def affineEncrypt(alpha, beta, input):
	output = ""
	for letter in input:
		in_index = string.ascii_lowercase.index(letter)
		out_index = ( in_index * alpha + beta) % 26 
		output += string.ascii_lowercase[out_index]
	return output

def main():
	encrypt = True
	alpha = 7
	beta = 22
	if encrypt:
		print("Starting encrypt with alpha = {0} and beta = {1}: \n".format(alpha, beta))
		input = "firstthesentenceandthentheevidencesaidthequeen"
		output = affineEncrypt(alpha, beta, input)
	else:
		print("Starting decrypt with alpha = {0} and beta = {1}: \n".format(alpha, beta))
		input = "falszztysyjzyjkywjrztyjztyynaryjkyswarztyegyyj"
		output = affineDecrypt(ModularInverse(alpha), beta, input)
	
	print( "The output is: \n" + output )

if __name__ == "__main__":
	main()
\end{lstlisting}
\end{minipage}

\end{document}
