\documentclass{article}
\usepackage{graphicx}
\usepackage{alltt}
\usepackage{amsmath}
\usepackage{amsfonts}
\usepackage{bigstrut}
\usepackage{enumerate}
\usepackage{fancyhdr}
\usepackage[top=.75in, bottom=.95in, left=.75in, right=.75in]{geometry}
\usepackage{float}
\usepackage{lastpage}
\usepackage{tikz}
\usepackage[latin1]{inputenc}
\usepackage{color}
\usepackage{array}
\usepackage{longtable}
\usepackage{calc}
\usepackage{multirow}
\usepackage{hhline}
\usepackage{ifthen}
\usepackage{listings}
\usepackage{circuitikz}
\usepackage{tabto}
%\lstset{language=VHDL,basicstyle=\ttfamily}
\definecolor{mygreen}{rgb}{0,0.6,0}
\definecolor{mygray}{rgb}{0.5,0.5,0.5}
\definecolor{mymauve}{rgb}{0.58,0,0.82}
\lstset{ %
  backgroundcolor=\color{white},   % choose the background color; you must add \usepackage{color} or \usepackage{xcolor}; should come as last argument
  basicstyle=\normalsize,        % the size of the fonts that are used for the code
  breakatwhitespace=false,         % sets if automatic breaks should only happen at whitespace
  breaklines=true,                 % sets automatic line breaking
  captionpos=b,                    % sets the caption-position to bottom
  commentstyle=\color{mygreen},    % comment style
  deletekeywords={...},            % if you want to delete keywords from the given language
  escapeinside={\%*}{*)},          % if you want to add LaTeX within your code
  extendedchars=true,              % lets you use non-ASCII characters; for 8-bits encodings only, does not work with UTF-8
  frame=single,	                   % adds a frame around the code
  keepspaces=true,                 % keeps spaces in text, useful for keeping indentation of code (possibly needs columns=flexible)
  keywordstyle=\color{blue},       % keyword style
  language=VHDL,                   % the language of the code
  morekeywords={*,...},            % if you want to add more keywords to the set
  numbers=left,                    % where to put the line-numbers; possible values are (none, left, right)
  numbersep=5pt,                   % how far the line-numbers are from the code
  numberstyle=\tiny\color{mygray}, % the style that is used for the line-numbers
  rulecolor=\color{black},         % if not set, the frame-color may be changed on line-breaks within not-black text (e.g. comments (green here))
  showspaces=false,                % show spaces everywhere adding particular underscores; it overrides 'showstringspaces'
  showstringspaces=false,          % underline spaces within strings only
  showtabs=false,                  % show tabs within strings adding particular underscores
  stepnumber=2,                    % the step between two line-numbers. If it's 1, each line will be numbered
  stringstyle=\color{mymauve},     % string literal style
  tabsize=2,	                   % sets default tabsize to 2 spaces
  title=\lstname                   % show the filename of files included with \lstinputlisting; also try caption instead of title
}
\floatstyle{plain}
\restylefloat{figure}
\pagestyle{fancy}
\fancyhead{}
\fancyfoot{}
\setlength{\headheight}{59.0pt}
\def\inputGnumericTable{}
\fancyhead[CO]{\textbf{Air Force Institute of Technology\\Department of Electrical and Computer Engineering\\
Data Security(CSCE 544)}\\ Homework \#1\\
 Due Date: 22-April-2019}
\lhead{\today}
\rhead{Page \thepage{} of \pageref{LastPage} }
\newlength\tindent
\setlength{\tindent}{\parindent}
\setlength{\parindent}{0pt}
\renewcommand{\indent}{\hspace*{\tindent}}

\usepackage[style=numeric, sorting=none]{biblatex}
\bibliography{myreferences.bib}

\begin{document}

\section*{You are able to place a probe at the output of a Linear Feedback Shift Register (LFSR) and observe the first 128 bits output.}


\begin{table}[H]
\centering
\caption{Input Table Format}
\begin{tabular}{lcccccccc}
                                  & \multicolumn{8}{c}{\textbf{Bit}}                                                                                                                                                                      \\
\textbf{Byte}                     & \multicolumn{1}{l}{1}  & \multicolumn{1}{l}{2}  & \multicolumn{1}{l}{3}  & \multicolumn{1}{l}{4}  & \multicolumn{1}{l}{5}  & \multicolumn{1}{l}{6}  & \multicolumn{1}{l}{7}  & \multicolumn{1}{l}{8}  \\ \cline{2-9} 
\multicolumn{1}{l|}{\textbf{0x0}} & \multicolumn{1}{c|}{0} & \multicolumn{1}{c|}{1} & \multicolumn{1}{c|}{0} & \multicolumn{1}{c|}{0} & \multicolumn{1}{c|}{1} & \multicolumn{1}{c|}{1} & \multicolumn{1}{c|}{1} & \multicolumn{1}{c|}{0} \\ \cline{2-9} 
\multicolumn{1}{l|}{\textbf{0x1}} & \multicolumn{1}{c|}{0} & \multicolumn{1}{c|}{0} & \multicolumn{1}{c|}{1} & \multicolumn{1}{c|}{0} & \multicolumn{1}{c|}{1} & \multicolumn{1}{c|}{1} & \multicolumn{1}{c|}{1} & \multicolumn{1}{c|}{1} \\ \cline{2-9} 
\multicolumn{1}{l|}{\textbf{0x2}} & \multicolumn{1}{c|}{0} & \multicolumn{1}{c|}{0} & \multicolumn{1}{c|}{1} & \multicolumn{1}{c|}{0} & \multicolumn{1}{c|}{1} & \multicolumn{1}{c|}{0} & \multicolumn{1}{c|}{0} & \multicolumn{1}{c|}{0} \\ \cline{2-9} 
\multicolumn{1}{l|}{\textbf{0x3}} & \multicolumn{1}{c|}{1} & \multicolumn{1}{c|}{1} & \multicolumn{1}{c|}{0} & \multicolumn{1}{c|}{0} & \multicolumn{1}{c|}{0} & \multicolumn{1}{c|}{0} & \multicolumn{1}{c|}{1} & \multicolumn{1}{c|}{0} \\ \cline{2-9} 
\multicolumn{1}{l|}{\textbf{0x4}} & \multicolumn{1}{c|}{0} & \multicolumn{1}{c|}{0} & \multicolumn{1}{c|}{0} & \multicolumn{1}{c|}{0} & \multicolumn{1}{c|}{1} & \multicolumn{1}{c|}{1} & \multicolumn{1}{c|}{1} & \multicolumn{1}{c|}{1} \\ \cline{2-9} 
\multicolumn{1}{l|}{\textbf{0x5}} & \multicolumn{1}{c|}{1} & \multicolumn{1}{c|}{1} & \multicolumn{1}{c|}{0} & \multicolumn{1}{c|}{1} & \multicolumn{1}{c|}{0} & \multicolumn{1}{c|}{1} & \multicolumn{1}{c|}{0} & \multicolumn{1}{c|}{1} \\ \cline{2-9} 
\multicolumn{1}{l|}{\textbf{0x6}} & \multicolumn{1}{c|}{1} & \multicolumn{1}{c|}{0} & \multicolumn{1}{c|}{0} & \multicolumn{1}{c|}{1} & \multicolumn{1}{c|}{1} & \multicolumn{1}{c|}{0} & \multicolumn{1}{c|}{1} & \multicolumn{1}{c|}{1} \\ \cline{2-9} 
\multicolumn{1}{l|}{\textbf{0x7}} & \multicolumn{1}{c|}{1} & \multicolumn{1}{c|}{0} & \multicolumn{1}{c|}{1} & \multicolumn{1}{c|}{1} & \multicolumn{1}{c|}{0} & \multicolumn{1}{c|}{1} & \multicolumn{1}{c|}{0} & \multicolumn{1}{c|}{0} \\ \cline{2-9} 
\multicolumn{1}{l|}{\textbf{0x8}} & \multicolumn{1}{c|}{1} & \multicolumn{1}{c|}{0} & \multicolumn{1}{c|}{0} & \multicolumn{1}{c|}{1} & \multicolumn{1}{c|}{1} & \multicolumn{1}{c|}{1} & \multicolumn{1}{c|}{0} & \multicolumn{1}{c|}{0} \\ \cline{2-9} 
\multicolumn{1}{l|}{\textbf{0x9}} & \multicolumn{1}{c|}{0} & \multicolumn{1}{c|}{1} & \multicolumn{1}{c|}{0} & \multicolumn{1}{c|}{1} & \multicolumn{1}{c|}{1} & \multicolumn{1}{c|}{1} & \multicolumn{1}{c|}{1} & \multicolumn{1}{c|}{0} \\ \cline{2-9} 
\multicolumn{1}{l|}{\textbf{0xA}} & \multicolumn{1}{c|}{0} & \multicolumn{1}{c|}{1} & \multicolumn{1}{c|}{0} & \multicolumn{1}{c|}{1} & \multicolumn{1}{c|}{0} & \multicolumn{1}{c|}{0} & \multicolumn{1}{c|}{0} & \multicolumn{1}{c|}{1} \\ \cline{2-9} 
\multicolumn{1}{l|}{\textbf{0xB}} & \multicolumn{1}{c|}{1} & \multicolumn{1}{c|}{0} & \multicolumn{1}{c|}{0} & \multicolumn{1}{c|}{0} & \multicolumn{1}{c|}{0} & \multicolumn{1}{c|}{1} & \multicolumn{1}{c|}{0} & \multicolumn{1}{c|}{0} \\ \cline{2-9} 
\multicolumn{1}{l|}{\textbf{0xC}} & \multicolumn{1}{c|}{0} & \multicolumn{1}{c|}{0} & \multicolumn{1}{c|}{0} & \multicolumn{1}{c|}{1} & \multicolumn{1}{c|}{1} & \multicolumn{1}{c|}{1} & \multicolumn{1}{c|}{1} & \multicolumn{1}{c|}{1} \\ \cline{2-9} 
\multicolumn{1}{l|}{\textbf{0xD}} & \multicolumn{1}{c|}{1} & \multicolumn{1}{c|}{0} & \multicolumn{1}{c|}{1} & \multicolumn{1}{c|}{0} & \multicolumn{1}{c|}{1} & \multicolumn{1}{c|}{0} & \multicolumn{1}{c|}{1} & \multicolumn{1}{c|}{1} \\ \cline{2-9} 
\multicolumn{1}{l|}{\textbf{0xE}} & \multicolumn{1}{c|}{0} & \multicolumn{1}{c|}{0} & \multicolumn{1}{c|}{1} & \multicolumn{1}{c|}{1} & \multicolumn{1}{c|}{0} & \multicolumn{1}{c|}{1} & \multicolumn{1}{c|}{1} & \multicolumn{1}{c|}{1} \\ \cline{2-9} 
\multicolumn{1}{l|}{\textbf{0xF}} & \multicolumn{1}{c|}{0} & \multicolumn{1}{c|}{1} & \multicolumn{1}{c|}{1} & \multicolumn{1}{c|}{0} & \multicolumn{1}{c|}{1} & \multicolumn{1}{c|}{0} & \multicolumn{1}{c|}{0} & \multicolumn{1}{c|}{1} \\ \cline{2-9} 
\end{tabular}
\end{table}
\begin{table}[H]
   \centering
   \caption{128 Bits Output Vector Format}
   \begin{verbatim}
0	1	0	0	1	1	1	0	0	0	1	0	1	1	1	1	0	0	1	0	1	0	0	0	1	1	0	0	0	0	1	0	0	0	0	0	1	1	1	1	1	1	0	1	0	1	0	1	1	0	0	1	1	0	1	1	1	0	1	1	0	1	0	0	1	0	0	1	1	1	0	0	0	1	0	1	1	1	1	0	0	1	0	1	0	0	0	1	1	0	0	0	0	1	0	0	0	0	0	1	1	1	1	1	1	0	1	0	1	0	1	1	0	0	1	1	0	1	1	1	0	1	1	0	1	0	0	1
   \end{verbatim} 
\end{table}
\subsection*{Encrypt the plaintext ``\textit{Hope this Helps!}'' by XORing the output of the LFSR and the plaintext. Assume 8-bits ASCII encoding for the plaintext.}
The output is shown below, in Hex:

	\begin{verbatim}
	06 40 58 a7 2f a1 f3 dd ef 7e 19 e1 73 db 44 48
	\end{verbatim}
\newpage
\subsection*{Explain the differences between the two different types of LFSR implementation: internal feedback, external feedback.}

An internal feedback implementation keeps the XOR gates between flip flops in the LFSR.
An external feedback sends all of the XOR outputs back to the input bit of the LFSR.
For an internal feedback LFSR, the next input bit will be the bit that was just observed at the output; while an external feedback LFSR's input is determined by the output of the taps.

\subsection*{Determine the LFSR's output for the next two bytes.}
The next two bytes are shown below, which is simply a continuation of the repeating sequence.

\begin{verbatim}
0	0	1	1	1	0	0	0
1	0	1	1	1	1	0	0
\end{verbatim}

\subsection*{Determine the LFSR's degree of polynomial and initial value.}
Based on the knowledge that the polynomial is a maximum length sequence, I know it is a 6-bit LFSR because the sequence repeated after 63 bits.
\begin{align*}
\text{Length } &= 2^m - 1 \\
		63 		&= 2^m - 1 \\
		64		&= 2^m \\
		m		&= 6
\end{align*}

The initial value was 
\begin{verbatim}
0	1	1	0	1	0
\end{verbatim}

\subsection*{Determine the LFSR's characteristics polynomial.}
The characteristic polynomial $P(x)$ is shown below:
\begin{equation}
P(x) = x^6 + x^5 + 1
\end{equation}


\end{document}
