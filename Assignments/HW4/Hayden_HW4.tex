\documentclass{article}
\usepackage{graphicx}
\usepackage{alltt}
\usepackage{amsmath}
\usepackage{amsfonts}
\usepackage{bigstrut}
\usepackage{enumerate}
\usepackage{fancyhdr}
\usepackage[top=.75in, bottom=.95in, left=.75in, right=.75in]{geometry}
\usepackage{float}
\usepackage{lastpage}
\usepackage{tikz}
\usepackage[latin1]{inputenc}
\usepackage{color}
\usepackage{array}
\usepackage{longtable}
\usepackage{calc}
\usepackage{multirow}
\usepackage{hhline}
\usepackage{ifthen}
\usepackage{listings}
\usepackage{circuitikz}
%\usepackage{tabto}
%\lstset{language=VHDL,basicstyle=\ttfamily}
\definecolor{mygreen}{rgb}{0,0.6,0}
\definecolor{mygray}{rgb}{0.5,0.5,0.5}
\definecolor{mymauve}{rgb}{0.58,0,0.82}
\lstset{ %
  backgroundcolor=\color{white},   % choose the background color; you must add \usepackage{color} or \usepackage{xcolor}; should come as last argument
  basicstyle=\normalsize,        % the size of the fonts that are used for the code
  breakatwhitespace=false,         % sets if automatic breaks should only happen at whitespace
  breaklines=true,                 % sets automatic line breaking
  captionpos=b,                    % sets the caption-position to bottom
  commentstyle=\color{mygreen},    % comment style
  deletekeywords={...},            % if you want to delete keywords from the given language
  escapeinside={\%*}{*)},          % if you want to add LaTeX within your code
  extendedchars=true,              % lets you use non-ASCII characters; for 8-bits encodings only, does not work with UTF-8
  frame=single,	                   % adds a frame around the code
  keepspaces=true,                 % keeps spaces in text, useful for keeping indentation of code (possibly needs columns=flexible)
  keywordstyle=\color{blue},       % keyword style
  language=VHDL,                   % the language of the code
  morekeywords={*,...},            % if you want to add more keywords to the set
  numbers=left,                    % where to put the line-numbers; possible values are (none, left, right)
  numbersep=5pt,                   % how far the line-numbers are from the code
  numberstyle=\tiny\color{mygray}, % the style that is used for the line-numbers
  rulecolor=\color{black},         % if not set, the frame-color may be changed on line-breaks within not-black text (e.g. comments (green here))
  showspaces=false,                % show spaces everywhere adding particular underscores; it overrides 'showstringspaces'
  showstringspaces=false,          % underline spaces within strings only
  showtabs=false,                  % show tabs within strings adding particular underscores
  stepnumber=2,                    % the step between two line-numbers. If it's 1, each line will be numbered
  stringstyle=\color{mymauve},     % string literal style
  tabsize=2,	                   % sets default tabsize to 2 spaces
  title=\lstname                   % show the filename of files included with \lstinputlisting; also try caption instead of title
}
\floatstyle{plain}
\restylefloat{figure}
\pagestyle{fancy}
\fancyhead{}
\fancyfoot{}
\setlength{\headheight}{59.0pt}
\def\inputGnumericTable{}
\fancyhead[CO]{\textbf{Air Force Institute of Technology\\Department of Electrical and Computer Engineering\\
Data Security(CSCE 544)}\\ Homework \#4\\Micah Hayden \\
 Due Date: \textbf{08-May-2019}}
\lhead{\today}
\rhead{Page \thepage{} of \pageref{LastPage} }
\newlength\tindent
\setlength{\tindent}{\parindent}
\setlength{\parindent}{0pt}
\renewcommand{\indent}{\hspace*{\tindent}}
\begin{document}

\section*{You intercept the following ciphertext generated using the following RSA public-key: pk=\{e,n\}=\{23,20413\}}
\subsection*{Determine the prime numbers $p$ and $q$:}
I utilized the following script to determine $p$ and $q$:
\lstinputlisting[language=Python]{findprimes.py}
Thus: 
$p = 137$, and $q = 149$.

\subsection*{Determine Euler's totient function $\phi(n)$:}
I calculated $\phi(n)$ as follows:
\begin{align*}
\phi(n) &= (p-1) \cdot (q-1) \\
\phi(n) &= (137 - 1) \cdot (149 - 1) \\
\phi(n) &= 20,128
\end{align*}

\subsection*{Determine the private-key=\{d,n\}:}
To compute $d$, the following relationship must hold:
\begin{equation}
d \cdot e \equiv 1 \mod{\phi(n)} 
\end{equation}
Knowing that $e=23$, and $\phi(n) = 20,128$, I calculated $d= 13127$:\footnote{Website used for calculation:  https://www.cs.drexel.edu/~jpopyack/IntroCS/HW/RSAWorksheet.html}
\begin{align*}
e \cdot d &= 23 \cdot 13127 \\
e \cdot d &= 301921 \\
			&= 301921 \mod{20128} = 1
\end{align*}

\newpage

\subsection*{Compute the plaintext for EACH of the following ciphertext :}
\subsubsection*{\{236, 2743, 7983, 5919, 20213, 5520, 19563, 17083, 17083, 19326, 5919, 17258, 5919, 17215, 19563, 20213, 4940, 496\}}

The plaintext is shown below, in Hex:
\{41, 6E, 64, 20, 73, 74, 69, 6c, 6c, 2c, 20, 49, 20, 72, 69, 73, 65, 2e\}\footnote{Calculated using the following website: http://extranet.cryptomathic.com/rsacalc/index}

\subsection*{Determine the ENGLISH plaintext:}
I converted the Hex values above into ASCII characters, resulting in the following English:

\textbf{And still, I rise.}

\end{document}
